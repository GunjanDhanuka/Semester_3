\documentclass{article}
\usepackage{amsmath}
%using hyperref package to create links as visible in the image
\usepackage{hyperref}
\hypersetup{colorlinks=true, linkcolor= blue, urlcolor=blue, citecolor=blue}
\title{Hello World!}
\author{Gunjan Dhanuka}
\date{November 15, 2021}
\begin{document}
    \maketitle %renders the title
    \thispagestyle{empty}
    \section{Getting Started}
    
        \textbf{Hello World!} Today I am learning \LaTeX. \LaTeX{} is a great program
        for writing math. I can write in line math such as $a^2 + b^2 = c^2$. I can also give
        equations their own space:
        
        \begin{equation}
        \gamma^2+\theta^2=\omega^2
        \end{equation}

        ``Maxwell's equations'' are named for James Clark Maxwell and are as follow:
        \begin{align} %Using align to align equations by the equal sign and the text
            %LHS
            \vec{\nabla} \cdot \vec{E} \quad&=
            %RHS 
            \quad\frac{\rho}{\epsilon_0}&&\text{Gauss's Law}\label{eq: GL}\\
            %LHS
            \vec{\nabla} \cdot \vec{B} \quad&=
            %RHS
            \quad0&&\text{Gauss's Law for Magnetism}\label{eq: GLM}\\
            %LHS
            \vec{\nabla} \times \vec{E} \quad&= 
            %RHS
            \quad-\frac{\partial\vec{B}}{\partial t}&&\text{Faraday's Law of Induction}\label{eq: FLI}\\
            %LHS
            \vec{\nabla} \times \vec{B} \quad&=
            %RHS
            \quad \mu_0\left(\epsilon_0\frac{\partial\vec{E}}{\partial t}+\vec{J}\right)&&\text{Ampere's Circuital Law}\label{eq: AL}
        \end{align}
        Equations \ref{eq: GL}, \ref{eq: GLM}, \ref{eq: FLI} and \ref{eq: AL} are some of the most important in Physics.
    
    \section{What about Matrix Equations?}
        \begin{equation*}
            \begin{pmatrix}
                a_{11}&&a_{12}&&\dots&&a_{1n}\\
                a_{21}&&a_{22}&&\dots&&a_{2n}\\
                \vdots&&\vdots&&\ddots&&\vdots\\
                a_{n1}&&a_{n2}&&\dots&&a_{nn}
            \end{pmatrix}
            \begin{bmatrix}
                v_1\\v_2\\\vdots\\v_n
            \end{bmatrix}
            =
            \begin{matrix}
                w_1\\w_2\\\vdots\\w_n
            \end{matrix}
        \end{equation*}
\end{document}