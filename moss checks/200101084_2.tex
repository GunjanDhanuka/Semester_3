%20001084
%Programming assignment 4
%Cs242

%Specifying the document class as article
\documentclass{article} 

%Specifyinh which packages to use
\usepackage{amsmath}\usepackage{hyperref}\hypersetup{colorlinks=true, linkcolor=blue, urlcolor=blue, citecolor=blue}
%For removing the page no from the bottom
\thispagestyle{empty}


%Specifying the beginning of the document
\begin{document}
%The first equation involving three integral signs
%LHS
\begin{equation*}\iiint\limits_V f(x,y,z)\,dV
 =
%RHS 
F\end{equation*}
\notag



%The second equation with the formula for a derivative of x w.r.t. y
%LHS
\begin{equation*}\frac{dx}{dy}
=
x'
=
%RHS 
\lim_{h \to 0}\frac{f(x+h)-f(x)}{h}\end{equation*}
\notag



%Third equation which defines modulus of x
%LHS
\begin{equation*}|x|
=
%RHS 
\begin{cases}-x, & \text{if $x < 0$}\\x, & \text{if $x \geq 0$}\end{cases}\end{equation*}
\notag



%fourth equation involving definition of a function F(x)
%LHS
\begin{equation*}F(x)=
%RHS 
 A_0 + \sum_{n=1}^N\left[ A_n\cos{\left(\frac{2\pi nx}{P}\right)}+B_n\sin{\left(\frac{2\pi nx}{P}\right)}\right]\end{equation*}
\notag



%Fifth equation with the hormula of summation
%LHS
\begin{equation*}\sum_n \frac{1}{n^s}
%RHS 
=\prod_p \frac{1}{1-\frac{1}{p^s}}\end{equation*}
\notag




\begin{equation*}         % equation* suppress equation numbering same for align*
%LHS 
m\ddot{x}+c\dot{x}+kx
=
%RHS 
F_0\sin(2\pi ft)\end{equation*}\notag



%Quadratic equations
%LHS
\begin{align*}f(x)\quad &=\quad x^2 + 3x + 5x^2 +8 +6x\\&=\quad 6x^2 +9x +8\\&=\quad x(6x+9)+8
\end{align*}


%LHS
$$X
=
%RHS 
\frac{F_0}{k}\frac{1}{\sqrt{(1-r^2)^2+(2\zeta r)^2}}$$\\\notag



\begin{equation*}G_{\mu\nu} \equiv R_{\mu\nu}-\frac{1}{2}Rg_{\mu\nu}=\frac{8\pi G}{c^4}T_{\mu\nu}\end{equation*}\notag


%Equation for the formig of glucose
%LHS
$$\mathrm{6CO_2+6H_2O \to C_6H_{12}O_6+6O_2}$$\notag


%Eqn for the formation of barium sulphate
%LHS
$$\mathrm{SO_4^{2-}+Ba^{2+} \to BaSO_4 }$$
\notag

%Matrix equation
\begin{equation*}
%First matrix
\begin{pmatrix}a_{11}&a_{12}&\dots&a_{1n}\\a_{21}&a_{22}&\dots&a_{2n}\\\vdots&\vdots&\ddots&\vdots\\a_{n1}&a_{n2}&\dots&a_{nn}
\notag
\end{pmatrix}
%Second matrix
\begin{pmatrix}v_{1}\\
v_{2}\\\vdots\\v_{n}
\notag
\end{pmatrix}
=
\begin{pmatrix}
%RHS Matrix
w_{1}\\w_{2}\\\vdots\\
w_{n}
\notag
\end{pmatrix}
\end{equation*}
\\


\begin{equation*}\frac{\partial{\bf{u}}}{\partial{t}}+(\bf{u}\cdot\nabla)\bf{u}-\nu\nabla^2\bf(u)=-\nabla h\notag
\end{equation*}



\[             % This is preferred to the $$ environment 
\alpha A \beta B \gamma \Gamma \delta \Delta \pi \Pi \omega \Omega \notag
\]  


\end{document}
%Made by Pranjal Singh